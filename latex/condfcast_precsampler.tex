\documentclass[notitlepage,a4paper,12pt]{article}
\usepackage[authoryear]{natbib}
\usepackage{amsmath}

\usepackage{amsfonts} 
\DeclareMathSymbol{\shortminus}{\mathbin}{AMSa}{"39}
\newcommand{\transpose}[1]{{#1}^{\sf{T}}}

\begin{document}

\title{Precision-based algorithms for conditional forecasting and scenario analysis in large state space models}

\author{
  Philipp Hauber\\
  Kiel Insitute for the World Economy
}
\date{\today}

\maketitle

\section{Introduction}

Conditional forecasting is important \citep{bgl_2015ijf}

Precision-based sampling algorithms have been proposed in the literature as an alternative to Kalman-filter based methods \citep{chanjeliazkov_2009}.

This paper proposes precision-based sampling algorithms for conditional forecasting and scenario analysis as an alternative to Kalman-filter based approaches. In an empirical application involving large Bayesian dynamic factor models and vector autoregressions I demonstrate its usefulness. 

\section{Precision-based sampling algorithms for conditional forecasting}

\subsection{Precision-based sampling}

Let $\mathbf{y_t} = [y_{1t}, \dots, y_{Nt}]'$ denote an $N \times 1$ vector of observations and $\mathbf{f_t}$ an $R \times 1$ vector of unobserved states. Consider the following state space model $\forall \: t = 1:T$

\begin{align}
    \mathbf{y_t} &= \Lambda \, \mathbf{f_t} + \epsilon_t \\
    \mathbf{f_t} &= \Phi \, \mathbf{f_{t-1}} + \upsilon_t
\end{align}

where 
$$
\begin{bmatrix}
    \epsilon_t \\
    \upsilon_t
\end{bmatrix}
\sim \mathcal{N}(0,
\begin{bmatrix}
\Sigma_{\epsilon} & \mathbf{0} \\
\mathbf{0} & \Sigma_{\epsilon}
\end{bmatrix}
)
$$
and $\Sigma_{\epsilon} = \text{diag}(\sigma^2_{\epsilon, 1}, \dots, \sigma^2_{\epsilon, N})$. The recursions are initialized with $\mathbf{f}_1 \sim \mathcal{N}(0, \Sigma_0)$.

Bayesian estimation of such models requires a draw from the distribution of the state vector conditional on the observations $\mathbf{y}_{1:T}$ and parameters $\Theta$, $p(\mathbf{f}_{1:T}|\mathbf{y}_{1:T}, \Theta)$. "Simulation smoothing" methods based on the Kalman filter/smoother have been proposed by Carter and Kohn (1994) and Durbin and Koopman (2001) and are widely used in the literature.

Precision-based sampling algorithms \citep{chanjeliazkov_2009} provide an alternative to these methods. To derive the sampler, we stack the observations over time $\mathbf{y} = [\mathbf{y}'_1, \dots, \mathbf{y}'_T]'$ which yields the following measurement equation

\begin{equation}
\mathbf{y} = \mathbf{L} \, \mathbf{f} + \epsilon
\end{equation} 

where $ \mathbf{L} = I_T \otimes \Lambda$ and $\epsilon \sim \mathcal{N}(0, I_T \otimes \Sigma_{\epsilon})$. The transition equation of the states can be written compactly as $\mathbf{H}_{\Phi} \, \mathbf{f_t} = \upsilon$ using the differencing matrix 
$$
\mathbf{H}_{\Phi}
=
\begin{bmatrix}
    I_R & 0_R & \dots \\
    \shortminus\Phi & I_R & \ddots \\
    0_R & \shortminus\Phi & I_R & \ddots \\
    \vdots & \ddots & \ddots & \ddots \\
    & & & \shortminus\Phi & I_R
\end{bmatrix}.
$$

The variance of the states (conditional on parameters) is thus given by $\mathcal{N}(\mathbf{0},\mathbf{H}^{-1}_{\Phi} \mathbf{V}_{\upsilon} \mathbf{H}^{-1'}_{\Phi})$ where 
$$
\mathbf{V}_{\upsilon} = 
\begin{bmatrix}
    \Sigma_{0} & 0_R & \dots \\
    0_R & \Sigma_{\upsilon} & \ddots \\
    \vdots & \ddots & \ddots \\
    & & & \Sigma_{\upsilon} 
\end{bmatrix}
.
$$

The precision matrix of the joint vector 

\begin{equation}
    \mathbf{z} = 
    \begin{bmatrix}
        \mathbf{f} \\
        \mathbf{y}
    \end{bmatrix}
    = 
    \begin{bmatrix}
        I_{TR} & 0 \\
        \mathbf{L} & I_{TN} 
    \end{bmatrix}
    \begin{bmatrix}
        \mathbf{f} \\
        \mathbf{\epsilon}
    \end{bmatrix} 
\end{equation}

is given by

\begin{align*}
    \mathbf{P}_z 
    &= 
    \mathbf{V}^{-1}_z \\
    &= 
    Var\left(
    \begin{bmatrix}
        \mathbf{f} \\
        \mathbf{y}
    \end{bmatrix}
    \right)^{-1} \\
    &=  
    \left(
    \begin{bmatrix}
        I & 0 \\
        \mathbf{L} & I
    \end{bmatrix}
    \begin{bmatrix}
        \mathbf{H}^{-1}_{\Phi} \mathbf{P}_{\upsilon} \transpose{\mathbf{H}^{-1}_{\Phi}} & 0 \\
        0 & \Sigma_{\epsilon}
    \end{bmatrix} 
    \begin{bmatrix}
        I & \mathbf{L}^{\sf{T}} \\
        0 & I
    \end{bmatrix}
    \right)^{-1}\\
    &=
    \begin{bmatrix}
        I & \shortminus\mathbf{L}' \\
        0 & I
    \end{bmatrix}
    \begin{bmatrix}
        \mathbf{H}'_{\Phi} \mathbf{P}^{-1}_{\upsilon} \mathbf{H}_{\Phi} & 0 \\
        0 & \Sigma^{-1}_{\epsilon}
    \end{bmatrix} 
    \begin{bmatrix}
        I & 0 \\
        \shortminus\mathbf{L} & I 
    \end{bmatrix} \\
    &=
    \begin{bmatrix}
        \mathbf{L}' \mathbf{H}'_{\Phi} \mathbf{P}^{-1}_{\upsilon} \mathbf{H}_{\Phi} L + \Sigma_{\epsilon} & \shortminus \mathbf{L}' \Sigma^{-1}_{\epsilon} \\
        \Sigma^{-1}_{\epsilon} L & \Sigma^{-1}_{\epsilon}
    \end{bmatrix} \\
    &= 
    \begin{bmatrix}
        \mathbf{P}_{\text{f}} & \mathbf{P}_{\text{fy}} \\
        \mathbf{P}'_{\text{fy}} & \mathbf{P}_{\text{y}}  
    \end{bmatrix}
\end{align*}

Using standard results for multivariate Normal distributions \citep[e.g.][pp. 86-87]{bishop_prml_2006}, we can derive the desired conditional distribution \citep[see also][eqn. 5]{chanjeliazkov_2009}:

\begin{align}
    p
    \left(
        \mathbf{f} \mid \mathbf{y}, \Theta
     \right) 
    &\sim \mathcal{N}
        \left(
            \mu_{\text{f}\mid y}, P_{\text{f}\mid y}^{-1}
        \right) \\
    &=    
    \mathcal{N}
        \left(
        \shortminus P^{-1}_{\text{f}} P_{\text{fy}} \mathbf{y}, P_{\text{f}}^{-1}
        \right) \\
    &= 
    \mathcal{N}
    \left(
    \shortminus P^{-1}_{\text{f}} L' \Sigma_{\epsilon}^{-1} \mathbf{y}, P_{\text{f}}^{-1}
    \right).
\end{align}

Draws from this distribution can be obtained efficiently since the $T\cdot R \times \cdot R$  $P_{\text{f}}$ is banded and its inversion not required. Specifically, the conditional mean can be calculated following \citet[][Algorithm 5]{rueheld_2005}, whereas a draw from the conditional distribution can be obtained in the following way: Let $v$ be a $T\cdot R \times 1$ matrix of independent draws from the standard Normal. Then solving $C'x = v$ for x via forward substitution where $C$ is the lower Cholesky factor of $P_{\text{f}}$  yields a draw from $\mathcal{N}(0, P^{-1}_{\text{f}})$ to which the mean $\mu_{\text{f}\mid y}$ can be added to obtain a draw from the desired conditional distribution \citep[][Algorithm 4]{rueheld_2005}.

\section{Empirical application}

\bibliographystyle{aernobold}
\bibliography{papers}

\end{document}